%
% As a work of the United States Government, this work is in
% the public domain within the United States. Additionally,
% the National Institute of Standards of Technology waives
% copyright and related rights in the work worldwide through the
% Creative Commons CC0 1.0 Universal Public Domain Dedication
% (https://creativecommons.org/publicdomain/zero/1.0/deed.en_US)
%
%

\subsection{Scope}

Simulating molecular systems that are interesting by today's standards, whether for biomolecular research, materials science, or a related field, is a challenging task.
However, computational scientists are often dazzled by the system-specific issues that emerge from such problems and fail to recognize that even ``simple'' simulations (e.g., alkanes) require significant care \cite{Schappals2017}.  In particular, questions often arise regarding the best way to adequately sample the desired phase-space or estimate uncertainties.  And while such questions are not unique to molecular modeling, their importance cannot be overstated: the usefulness of a simulated result ultimately hinges on being able to confidently and accurately report uncertainties along with any given prediction \cite{Nicholls2014}.  In the context of techniques such as molecular dynamics (MD) and Monte Carlo (MC), these considerations are especially important, given that even large-scale modern computing resources do not guarantee adequate sampling.

This article therefore aims to provide best-practices for reporting simulated observables, assessing confidence in simulations, and deriving uncertainty estimates (more colloquially, ``error bars'') based on a variety of statistical techniques applicable to physics-based sampling methods and their associated ``enhanced'' counterparts.  As a general rule, we advocate a tiered approach to computational modeling.  In particular, workflows should begin with back-of-the-envelope calculations to determine the feasibility of a given computation, followed by the actual simulation(s).  Semi-quantitative checks can then be used to check for adequate sampling and assess the quality of data.  Only once these steps have been performed should one actually construct estimates of observables and uncertainties.  In this way, modelers avoid unnecessary waste by continuously gauging the likelihood that subsequent steps will be successful.  Moreover, this approach can help to identify seemingly reasonable data that may have little value for prediction and/or be the result of a poorly run simulation.

It is worth emphasizing that in the last few years, many works have developed and advocated for uncertainty quantification (UQ) methods not traditionally used in the MD and MC communities.  In some cases, these methods buck trends that have become longstanding conventions, e.g., the practice of only using uncorrelated data to construct statistical estimates.  One goal of this manuscript is therefore to advocate newer UQ methods when these are demonstrably better.  Along these lines, we wish to remind the reader that better results are not only obtained from faster computers, but also by using data more thoughtfully.
It is also important to appreciate that debate continues even among professional statisticians on what analyses to perform and report \cite{Leek2017}.

The reader should be aware that there is not a ``one-size-fits-all'' approach to UQ.  Ultimately, we take the perspective that uncertainty quantification in its broadest sense aims to provide actionable information for making decisions, e.g., in an industrial research and development setting or in planning future academic studies.  A simulation protocol and subsequent analysis of its results should therefore take into account the intended audience and/or decisions to be made on the basis of the computation.  In some cases, quick-and-dirty workflows can indeed be useful if the goal is to only provide order-of-magnitude estimates of some quantity.  We also note that uncertainties can often be estimated through a variety of techniques, and there may not be consensus as to which, if any, are best.  {\it Thus, a critical component of any UQ analysis is communication, e.g., of the assumptions being made, the UQ tools used, and the way that results are interpreted.}  Educated decisions can only be made through an understanding of both the process of estimating uncertainty and its numerical results.

While UQ is a central topic of this manuscript, our scope is limited to issues associated with sampling and related uncertainty estimates.  We do not address systematic errors arising from inaccuracy of force-fields, the underlying model, or parametric choices such as the choice of a thermostat time-constant.  See, for example, Refs.~\cite{Leimkuhler,Rizzi2,Rizzi3,Rizzi4} for methods that address such problems.  Similarly, we did not address bugs and other implementation errors, which will generally introduce systematic errors. Finally, we do not consider model-form error and related issues that arise when comparing simulated predictions with experiment.  Rather, we take the raw trajectory data at face value, assuming that it is a valid description of the system of interest.\footnote{In more technical UQ language, we restrict our scope to {\it verification} of simulation results, as opposed to {\it validation}. Readers may consult \url{https://github.com/MobleyLab/basic_simulation_training} and \url{https://github.com/shirtsgroup/software-physical-validation} regarding foundations of molecular simulation and validation of simulation results, respectively.}
