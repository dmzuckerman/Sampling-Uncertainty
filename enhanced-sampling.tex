\section{Enhanced sampling [Write as supplements, which can later be broken out as separate documents?]}

While recent advances in computational hardware have allowed MD simulations of systems with biological relevance to routinely reach timescales ranging from hundreds of \si{\nano\second} to \si{\micro\second}, in many cases this is still not long enough to obtain equilibrated (i.e. Boltzmann-weighted) structural populations. Enhanced sampling methods can be used to obtain well-converged ensembles faster than conventional MD. In general, enhanced sampling methods work through a combination of modifying the underlying energy landscape to increase the rate at which energy barriers are crossed with some form of reweighting to recover the unbiased ensemble. However, their use does not automatically guarantee a converged ensemble, and care must be taken when using and evaluating enhanced sampling methods.

\subsection{Replica Exchange Molecular Dynamics}
Before running replica exchange MD (REMD)\cite{Sugita1999} one should ask the following questions:
\begin{itemize}
  \item Is REMD needed? Can the process be observed multiple times via conventional MD?
  \item Is dynamic information needed (e.g. diffusion, etc). This information is not easily extracted from REMD simulations.
  \item Is REMD computationally feasible? Typical implementations of REMD require at least 1 computational node per replica. In general, the number of replicas needed to get reasonable exchange rates is proportional to the square root of the number of degrees of freedom that vary between different replicas.
  \item What system variable(s) should be modified to increase sampling? Consider what the barriers to sampling are. Temperature is probably the most commonly used variable, but can be problematic: 1) temperature is only good for accelerating certain processes (e.g. protein unfolding but not folding), 2) temperature affects all degrees of freedom of a system and so typically requires large numbers of replicas, especially for explicitly solvated systems. It may be worthwhile to consider a different modification of the Hamiltonian, such as scaling dihedral force constants.
\end{itemize}

- Compare to standard sampling (e.g., straight MD) for a simple system
- Complex correlation structures in complex methods indicate comparison of multiple independent runs will be useful
    - Explain what to compare [Dan Z]
- Replica exchange [Daniel Roe]
    - Round-trips (necessary but not sufficient)
    - Compare ‘coordinate trajectories’ - distributions from temperature/Hamiltonian-wandering trajectories should match
    - Examine replica residence times
- Weighted ensemble (WE)
    - See this WE overview doc, particularly limitations section
    - Key concept: ‘Tree’ of trajectories generated by WE leads to strong correlations, requiring care
    - Key concept: WE simulation generically relaxes from the initial distribution toward the ultimate distribution which could be equilibrium (if no feedback/recycling or external driving) or a non-equilibrium steady state (if feedback from specified target to initial state)
    - Safest approach: Use multiple runs, which are fully independent.  Perform as many runs as needed to reduce the statistical uncertainty (std err of mean) for quantity of interest
    - For a single observable, the time course of the value can be analyzed using the usual methods of analyzing time-correlated data (see above - e.g., block-averaging)

