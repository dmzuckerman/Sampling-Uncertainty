\section{Quantification of Global Sampling}

With ideal trajectory data, one would hope to be able compute arbitrary observables with reasonably small error bars.
During a simulation, it is not uncommon to monitor specific observables of interest, but after the data is obtained, it may prove necessary to compute observables not previously considered.
These points motivate the goal of estimating global sampling quality, which can be framed most simply in the context of single-trajectory data:
``Among the very large number of simulation frames (snapshots), how many are statistically independent?''  
From a dynamical perspective, which also applies to Monte Carlo data, how long must one wait before the system completely loses memory of its prior configuration?

The discussion here will focus largely on biomolecular systems, or more precisely, on systems for which it is straightforward to define a meaningful scalar distance between configurations.

A key caveat is needed before proceeding.  
Analysis of trajectory data generally cannot make inferences about parts of configuration space not visited.
It is generally impossible to know whether configurational states absent from a trajectory are appropriately absent because they are highly improbable (extremely high energy) or because the simulation simply failed to visit them because of a high barrier or bad luck.


