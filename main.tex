%%%%%%%%%%%%%%%%%%%%%%%%%%%%%%%%%%%%%%%%%%%%%%%%%%%%%%%%%%%%
%%% LIVECOMS ARTICLE TEMPLATE
%%% ADAPTED FROM ELIFE ARTICLE TEMPLATE (8/10/2017)
%%%%%%%%%%%%%%%%%%%%%%%%%%%%%%%%%%%%%%%%%%%%%%%%%%%%%%%%%%%%
%%% PREAMBLE
\documentclass[9pt,bestpractices]{livecoms}
% Use the 'onehalfspacing' option for 1.5 line spacing
% Use the 'doublespacing' option for 2.0 line spacing
% use the 'lineno' option for adding line numbers.
% Please note that these options may affect formatting.

\usepackage[version=4]{mhchem}
\usepackage{siunitx}
\DeclareSIUnit\Molar{M}
\newcommand{\versionnumber}{1.0}  % you should update the minor version number in preprints and major version number of submissions.
%%%%%%%%%%%%%%%%%%%%%%%%%%%%%%%%%%%%%%%%%%%%%%%%%%%%%%%%%%%%
%%% ARTICLE SETUP
%%%%%%%%%%%%%%%%%%%%%%%%%%%%%%%%%%%%%%%%%%%%%%%%%%%%%%%%%%%%
\title{Best Practices for Quantification of Uncertainty and Sampling Quality in Molecular Simulations: v\versionnumber}

% Everyone should put in their name properly, institution, email
\author[1*\authfn{1}]{Alan Grossfield}
%\author[2*\authfn{1}]{Pascal T. Merz}
\author[2*\authfn{1}]{Paul N. Patrone}
\author[3*\authfn{1}]{Daniel R. Roe}
\author[4*\authfn{1}]{Andrew J. Schultz}
\author[5*\authfn{1}]{Daniel W. Siderius}
\author[6*\authfn{1}]{Daniel M. Zuckerman}
\affil[1]{University of Rochester Medical Center, Department of Biochemistry and Biophysics}
%\affil[2]{Department of Chemical and Biological Engineering, University of Colorado Boulder}
\affil[2]{Applied Computational and Mathematics Division, National Institute of Standards and Technology}
\affil[3]{Laboratory of Computational Biology, National Heart Lung and Blood Institute, National Institutes of Health}
\affil[4]{Department of Chemical and Biological Engineering, University at Buffalo, The State University of New York}
\affil[5]{Chemical Sciences Division, National Institute of Standards and Technology}
\affil[6]{Department of Biomedical Engineering, Oregon Health \& Science University}

%\corr{email1@example.com}{FMS}  % Correspondence emails.  FMS and FS are the appropriate authors initials.
\corr{alan_grossfield@urmc.rochester.edu}{AG}
%\corr{Pascal.Merz@colorado.edu}{PTM}
\corr{paul.patrone@nist.gov}{PNP}
\corr{daniel.roe@nih.gov}{DRR}
\corr{ajs42@buffalo.edu}{AJS}
\corr{daniel.siderius@nist.gov}{DWS}
\corr{zuckermd@ohsu.edu}{DMZ}

\contrib[\authfn{1}]{These authors contributed equally to this work.}

%% Common symbols
\newcommand{\stdunc}{s^*}


%\presentadd[\authfn{3}]{Department, Institute, Country}
%\presentadd[\authfn{4}]{Department, Institute, Country}

%%%%%%%%%%%%%%%%%%%%%%%%%%%%%%%%%%%%%%%%%%%%%%%%%%%%%%%%%%%%
%%% ARTICLE START
%%%%%%%%%%%%%%%%%%%%%%%%%%%%%%%%%%%%%%%%%%%%%%%%%%%%%%%%%%%%

\begin{document}

\maketitle

\begin{abstract}
The quantitative assessment of uncertainty and sampling quality is essential in molecular simulation.
Many systems of interest are highly complex, at the edge of current computational capacity, and simulators must understand and communicate statistical uncertainties so that `consumers' of the data understand the meaning and limitations of the simulation data.
This article covers key analyses appropriate for trajectory data generated by straightforward simulation methods such as molecular dynamics and (single Markov chain) Monte Carlo, as well as providing guidance for analyzing some `enhanced' sampling approaches.
We do not discuss \emph{systematic} errors arising from inaccuracy in the chosen model or force field.
\end{abstract}

\section{Introduction: Scope and definitions}
\label{sec:scope}
% \subsection{Scope}
\subsection{Scope}

Simulating molecular systems that are interesting by today's standards is a challenging task.  In addition to the various system-specific issues that modelers must address, questions often arise concerning, e.g. the best way to adequately sample the desired phase-space or estimate uncertainties.  And while these latter questions are not unique to molecular modeling, their importance cannot be overstated: the usefulness of a simulated result ultimately hinges on being able to confidently and accurately report uncertainties along with any given prediction.  

This article therefore aims to provide best-practices for reporting simulated observables, assessing confidence in simulations, and deriving uncertainty estimates (more colloquially, ``error bars'') based on a variety of statistical techniques developed (in part) for physics-based sampling methods -- e.g.\ molecular dynamics (MD) and Monte Carlo (MC) -- and their associated ``enhanced'' counterparts.  As a general rule, we advocate a tiered approach to modeling.  More specifically, workflows should begin with back-of-the-envelope calculations to determine the feasibility of a given computation, followed by the actual simulation(s).  Semi-quantitative checks can then be used to check for adequate sampling and assess the quality of data.  Only once these steps have been performed should one actually construct estimates of observables and uncertainties.  In this way, modelers avoid unncessary waste associated with poor planning and/or making predictions on the basis of ``junk'' data.

It is worth emphasizing that in the last few years, many works have developed data analysis and uncertainty quantification (UQ) methods not traditionally used in the MD and MC communities.  In some cases, these methods buck trends that have become longstanding conventions in certain communities, e.g.\ the practice of only using uncorrelated data to construct statistical estimates.  A key goal of this manuscript is therefore to advocate newer methods when these are demonstrably better.  Along these lines, we wish to remind the reader that better results are not only obtained from faster computers, but also by using data more thoughtfully.   

While UQ is a central topic of this manuscript, our scope is nonetheless limited to issues associated with sampling and related uncertainty estimates.  We do not address systematic errors arising from inaccuracy of force-fields, the underlying model, or parametric choices such as the choice of a thermostat time-constant.  See, for example, Refs.~\cite{Limecooler,Rizzi2,Rizzi3,Rizzi4} for methods that address such problems.  Moreover, we do not consider model-form error and related issues that arise when comparing simulated predictions with experiment.  Rather, we take the raw trajectory data at face value, assuming that it is a valid description of the system of interest.\footnote{In more technical UQ language, we restrict our scope to {\it verification} of simulation results, as opposed to {\it validation}.}

The rest of this manuscript is organized as follows: 

%PNP note: originally we had the phrase, "Some problems and systems may be better studied with cruder techniques and analyses, which will not be covered here."  This seems like an attempt to limit the scope, but I don't know what the original author had in mind.  What types of analyses are we ruling out versus keeping in?


%simulation studies attempting to quantify observables and derive reliable estimates of uncertainty (e.g., error bars) based on `standard' canonical sampling methods (e.g., molecular dynamics and Monte Carlo) and associated `enhanced' sampling methods.
%Some problems and systems may be better studied with cruder techniques and analyses, which will not be covered here.
%This article also will not cover issues of systematic error arising from inaccuracy in force field (underlying model) or even from the simulation setup.
%Rather, we will take raw trajectory data at face value, assuming it is a valid outcome given the underlying model.
%We emphatically will \emph{not} assume that a trajectory has been sufficiently `equilibrated' or 'converged'.

%PNP note:  The last sentence above is a bit misleading because we have a whole section on equilibrating.  It think it's better to drop 


%\subsection{Key Definitions [to be refined]}
\subsection{Key Definitions}

In order to make the discussion that follows more precise, we define key terms used in subsequent sections.
We caution that while many of these concepts are familiar, our terminology follows the {\it International Vocabulary of Metrology} (VIM)\citep{JCGM:VIM2012}, a standard that sometimes differs from the conventional or common language of engineering statistics. 
For additional information about or clarification of the statistical meaning of terms in the VIM, we suggest that readers consult the {\it Guide to the expression of uncertainty in measurement} (GUM)\citep{JCGM:GUM2008}.
For clarity, we highlight differences between conventional terms and the VIM in the section following the glossary.
In cases of lexical ambiguity, we hold to the definition of terms as given in the VIM.

%The reader should be familiar with a set of basic statistical and simulation concepts.
%[and we should provide links/refs for each.]

%NOTE FOR DAN S.  In some places I comment on definitions using ``Remarks'' that immediately follow the definition.  Not sure how well this format jibes with the rest of the doc.  Anyways, feel free to merge these remarks into the definitions if you think appropriate.

%NOTE FOR PAUL P. I've restructed the document a bit, applying the "one sentence per line" rule, which makes diff'ing the document easier.

% Reply to Dan S.: One sentence rule works for me

\subsubsection{Glossary of Statistical Terms}
\begin{itemize}
%terms added by PNP and DWS

\item {\bf Random quantity}: A quantity whose numerical value is inherently unknowable or unpredictable.

\smallskip

\textbf{\textit{Remark:}}  Informally speaking, we assume that a random quantity can be described in terms of a mean (or expected) value and a dispersion.  The latter characterizes the extent to which a {\it realization} of the random quantity differs from the mean.  More formally, we assume that the probability of a random variable $X$ taking value $x$ is given by $P(x)dx$, where $P(x)$ is a probably density and $dx$ is an infinitesimal width about $x$.

\smallskip

\textbf{\textit{Remark:}} Virtually all simulation tools (even those using quasi-random number generators) are deterministic.  As such, the output of any given simulation is never truly random, since it is the composition of knowable and predictable calculations.  However, it is generally infeasible to reproduce simulated calculations by hand, so that these are {\it in practice} unknowable.  Moreover, the chaotic nature of typical simulated systems leads to a situation in which the frequency of a given system configuration is well described by probability densities of statistical mechanics.  See Ref.~\cite{Leimkuhler} for more discussion of this rather deep point.

\item {\bf True value:}  The value of a quantity that is consistent with its definition and is the objective of an idealized measurement or simulation.

\smallskip
\textbf{\textit{Remark:}} Often the adjective ``true'' is dropped when reference to the definition is clear by context. \citep{JCGM:GUM2008,JCGM:VIM2012}
%Not sure that the last phrase is entirely consistent with GUM, although I'm not sure this is necessarily a problem.  To a certain extent, we may need to tailor the language of GUM/VIM to the audience at hand.
\smallskip

\textbf{\textit{Remark:}} Generally speaking, we use the descriptor ``true value'' in reference to quantities such as the expected value and standard deviation of a simulation output (were we able to run infinitely many simulations).
As these quantities are inherently unknowable, we can only estimate their values and associated uncertainties.

\item {\bf Standard Uncertainty}: Uncertainty in a result (e.g.\ prediction of a true value) as expressed in terms of a standard deviation.
\smallskip 

\textbf{\textit{Remark:}} The definition of standard uncertainty does not specify how to calculate the standard deviation.
This choice ultimately rests with the modeler and should be dictated by the details of the uncertainty relevant to the problem at hand.  

\item {\bf Arithmetic mean}: An estimate of the (true) expectation value of a random quantity. The arithmetic mean is given by the formula
  %
  \begin{equation}
    \bar{q} = \dfrac{1}{n} \sum_{k=1}^{n} q_k \label{def:arith_mean}
  \end{equation}
  %
  where $q_j$ is an experimental or simulated realization of the random variable and $n$ is the number of samples. 
\smallskip 
%PNP Note: I added "or simulated" after "experimental" in the line above

\textbf{\textit{Remark:}} It is straightforward to show that the arithmetic mean is a random quantity whose expectation value is that same as that of the $q_j$, provided the latter have a common mean.

\item {\bf Experimental standard deviation}: An estimate of the (true) standard deviation of a random variable, given by the formula
  % 
  \begin{equation}
    s\left(q_k\right) = \sqrt{\dfrac{\sum_{j=1}^n\left(q_j - \bar{q}\right)^2}{n-1}} \label{def:exp_st_dev}
  \end{equation}
  %
  where $q_j$, $\bar{q}$, and $n$ are as defined previously. 
  
\item {\bf Experimental standard deviation of the mean}: An estimate of the standard deviation of the distribution of the arithmetic mean, given by the formula
  % 
  \begin{equation}
    s\left(\bar{q}\right) = \dfrac{s\left(q_k\right)}{\sqrt{n}}. \label{def:exp_st_dev_mean}
  \end{equation}
  %
  \smallskip
  
\textbf{\textit{Remark:}} The experimental standard deviation of the mean characterizes the dispersion of the arithmetic mean relative to its expectation.
See \hyperref[def:exp_st_dev]{example link to definition of the experimental standard deviation}.
  
\item {\bf Precision}: The amount of variability in an estimate (e.g., based on repeating a given simulation protocol multiple times).  %Better sampling in an individual simulation leads to higher precision.  
%I commented out the last phrase because ``better'' seems a bit vague to me in this context.  Should this comment be placed elsewhere?  
  
  
\item {\bf Accuracy}: The degree to which a result agrees with a reference value.
The latter may be an experimental measurement or the result of a well-sampled simulation.  
  
\item {\bf Raw data}: The numbers that the computer program directly generates as it proceeds through a sequence of states [the phase-space trajectory for molecular dynamics (MD) or the Markov chain for Monte Carlo (MC)].
For example, a MC simulation generates a sequence of configurations, for which there are associated properties such as the instantaneous pressure, temperature, volume, etc.
% NOTE FOR DAN S.  Programs also directly compute things like pressure, temperature, volume, etc., which are necessary for controlling thermostats, barostats, etc.  Should we include these in the list of raw data?
% DWS REPLY: How is this modification?
% PNP REPLY: Overall fine.  I changed Monte Carlo to MD in the last sentence.  Do MC simulations have associated pressures?  I always thought that could only be computed in the context of a dynamical simulation.
% DWS REPLY2: I switched it back to MC - one of my objectives in this paper to make sure it is not too MD-centric. Aside: yes, you can calculate an instantaneous pressure in MC via the molecular virial if using a fixed-N ensemble.
  
\item {\bf Derived observables}: Quantities derived from `non-trivial' (and often non-linear) analyses of raw data, e.g., properties that may not be computed for a single configuration such as free energies.
  
\item {\bf Indepent observables}: Random quantities $x$ and $y$ for which the joint probability density can be written as a product of individual probability densities $P(x,y)=P(x)P(y)$.  
  
\item {\bf (Linearly) Uncorrelated observables}:  If quantities $q_j$ and $q_k$ have mean values $\left< q_j \right> $ and $\left< q_k \right>$, then $q_j$ and $q_k$ are linearly uncorrelated if
% 
\begin{equation}
  \left< \left(q_j - \left<q_j\right> \right) \left(q_k - \left<q_k\right> \right) \right>=0
\end{equation}
%
where $\left< \star \right>$ denotes the (true) expectation value.

\smallskip

\textbf{\textit{Remark:}} Independence of random variables implies that they are linearly uncorrelated.  The converse, however, is not true.  

\smallskip

\textbf{\textit{Remark:}} In practice, it is easier to (empirically) show that random variables are uncorrelated than independent.  However, linear correlations are often all that are needed to arrive at uncertainty estimates for various quantities.  See, e.g.\ the discussion of correlation times and autocorrelation analyses.
%PNP: where is that discussion?



\item {\bf Correlated observables}: Random quantities that are not independent.

\item {\bf Correlation time}: In time-series data of a random quantity $q(t)$ (e.g.\ a physical property from a MC or MD trajectory), this is the time $\tau$ over which $q(t)$ and $q(t+\tau)$ remain (linearly) correlated.

\smallskip

\textbf{\textit{Remark:}} Generally speaking, MC and MD trajectories generate new configurations from preceding ones.
Thus, the correlation time can be interpreted as the time over which the system retains memory of its previous states.
Such correlations are often {\bf stationary}, meaning that $\tau$ is independent of $t$.
Roughly speaking, the total simulation time divided by the longest correlation time yields an estimate of the number of {\it uncorrelated} samples generated by a simulation.


%The time over which realizations of a random quantity indexed by a time-series (e.g.\ a physical property from a Monte Carlo or molecular dynamics trajectory) retain ``memory'' of one another, since each configuration is generated from the preceding one. Roughly, the total simulation time divided by the (longest) correlation time gives an estimate of the number of \emph{independent} samples that will govern overall statistical quality of the data.

\item {\bf Two-sided confidence interval}: A statistically derived bound pair of lower and upper bounds between which the (true) expectation value of a random quantity is likely to fall, as quantified by a {\it confidence level} (typically given as a percentage, e.g., 95~\%). The confidence level, in conjunction with the sample size, determines the {\it coverage factor}, which is multiplied by \hyperref[def:exp_st_dev_mean]{$s\left(\bar{q}\right)$} to assign the bounds of the confidence interval.

% Older versions
%\item Precision: The amount of variability in an estimate (based on repeating a given simulation protocol multiple times).
%      Better sampling in an individual simulation leads to higher precision.
%      The standard error of the mean is usually the key measure of the \emph{scale} of the statistical uncertainty - i.e., %precision.
%    \item Confidence Interval: A statistically derived pair of minimum and maximum values within which the mean of an observable is likely to fall, as quantified by a percentage.  Note that useful confidence intervals (e.g., 90 or 95\%) tend to be roughly \emph{four times} the standard error of the mean (from minimum to maximum).
%    \item Accuracy: The degree of agreement with a reference value, which may be an experimental measurement or the result of a %well-sampled simulation.
%    \item Raw data: The numbers that the computer program directly generates as it runs -- typically configurations, and also %velocities in molecular dynamics.
%    \item Derived observables:  Quantities derived from `non-trivial' analyses of raw data, such as free energies.
%    \item Correlation time: The time over which samples/configurations in a MC or MD trajectory retain some "memory" of one another, since each configuration is generated from the preceding one.  Roughly, the total simulation time divided by the (longest) correlation time gives an estimate of the number of \emph{independent} samples that will govern overall statistical quality of the data.

\end{itemize}

\subsubsection{Discussion of terminology}

As surveyed by Refs.~\citep{JCGM:GUM2008,JCGM:VIM2012}, the discussion that originally motivated many of theses definitions is rather philosophical.
Nonetheless, it bears repeating here, if only to force a level of honesty regarding what we can actually hope to achieve with simulations.

The fundamental idea underpinning this discussion is the observation that true values are inherently unknowable.
As an added twist, the {\it Uncertainty Approach} advocated by Ref.~\citep{JCGM:GUM2008} also points out that the {\it definitions of true values may be imprecise,} so that the latter are in fact not even uniquely defined.\footnote{The GUM treats this definitional uncertainty as negligible compared to other sources.
However, the notion of definitional uncertainty is worth revisiting in the context of simulated data.
For example, we are aware of at least one situation in which the large scale of noise in simulated data (as compared with experiments) induces subjectivity in the definition of the glass-transition temperature, which ultimately affects uncertainties in a non-trivial way.
See Ref.~\citep{patrone1} and \textcolor{red}{Cite TG doc}.}
As such, assignments of ``error'' are impossible to make, since these are defined relative to the true value.
Rather, statistical analyses are better interpreted as quantifying our state of knowledge, e.g., our ability to confidently state that the the ``true'' mean is within some interval.

From a practical standpoint, this leads one to revisit definitions of uncertainty as defined, for example, by Eqs.~\ref{def:exp_st_dev} and \ref{def:exp_st_dev_mean}.
Conventionally this has gone by the name ``standard error,'' but in keeping with the perspective already laid out, we advocate the use of \hyperref[def:exp_st_dev_mean]{experimental standard deviation of the mean}.
To maintain consistency with the VIM and GUM, we also use ``sample mean'' and ``sample standard deviation'' with their counterparts ``arithmetic mean'' and ``experimental standard deviation.''
The reader should be aware that this older language is still used frequently throughout the literature.


%Table of equivalencies

%Arithmetic mean = ``sample mean''\\
%Experimental standard deviation: ``sample standard deviation''\\
%Experimental standard deviation of the mean: ``standard error''\\



\section{Best Practices Checklist}
\begin{enumerate}
\item
Pre-simulation sanity checks and planning tips: There is no guarantee that any method (enhanced or otherwise) can sample any given system
    \begin{itemize}
    \item See best-practices papers on simulation background and planning/setup [Link out to simulation background and preparation documents - github]
    \item Are system timescales known experimentally and feasible computationally based on published literature?
      - If timescales are too long for straight-ahead MD, is an enhanced method being used for which there are precedents for systems of similar complexity?
    \item Read a good article or book on sampling assessment (this one or a reference herein).  Understanding error is a technical endeavor.
      - Key concept: Connection between the equilibrium ensemble and individual trajectory (may or may not reach equilibrium); equilibrium vs. non-equilibrium.
    \item Consider multiple runs vs. single run.  Multiple runs may be especially useful in assessing uncertainty for enhanced sampling methods.
      - Make initial configuration as diverse as possible  … but note that if results depend on initial configs, that implies insufficient sampling (to be pedantic, it always does if sampling is finite, but it’s about figuring out variability and confidence intervals
      - Look for automated construction methods for reproducibility
    \item Check your code/method via a simple benchmark system.  [Link out to software validation doc]
    \end{itemize}
\item
Perform quick-and-dirty data checks which can rule out (but not ensure) sufficient sampling: Necessary vs. sufficient
    \begin{itemize}
    \item Look at time series -- think in advance about what states should exist. How many transitions do you see? If you have 1 transition, you can’t talk about populations
    \item Plot as many properties as you can think of, even if they’re not interesting
    \item Plot pairwise configurational distances (e.g., RMSD values for biomolecules) in greyscale for $\sim$100 evenly spaced frames
    \item Visualize the trajectory graphically -- look for slow motions.  BE SKEPTICAL!
    \item Compare observable different fractions of a run (DMZ thirds idea)
    \item Andrew: short vs. very short
    \item Daniel R: Compare runs from different initial conditions - be sure initial conditions are ‘different enough’
    \end{itemize}
\item
 Remove an ‘equilibration’/’burn in’/transient portion of a single MD or MC trajectory and perform analyses only on remaining ‘production’ portion of trajectory.  \emph{Underlying concept:} An initial configuration is unlikely to be representative of the desired ensemble and the resulting relaxation process must be accounted for.
\item
 Consider computing a quantitative measure of global sampling, particularly for a biomolecular system - i.e., attempt to estimate the number of statistically independent samples in a trajectory.  \emph{Underlying concept:} Trajectory 'frames'/configurations are highly correlated because one frame is generated from the preceding one, and estimating the degree of correlation is essential to understanding overall simulation quality.
\item
 Quantify uncertainty in specific observables of interest.  \emph{Underlying concept:} The statistical uncertainty in the estimate of an observable's \emph{average} decreases as more independent samples are obtained, and it can be much smaller than the standard deviation -- which measures the range of variation in the observable.
\item
 Use special care and uncertainty analyses for enhanced sampling methods.  \emph{Underlying concept:} The use of multiple, potentially correlated trajectories within a single enhanced-sampling simulation can invalidate the assumptions underpinning traditional analyses of uncertainty.
\end{enumerate}

% we will have a bunch of includes here
\section{Pre-simulation ``sanity checks'' and planning tips}
\label{sec:sanity}

Sampling a molecular system that is complex enough to be ``interesting'' in modern science is often extremely challenging, and similar difficulties apply to studies of ``simple'' systems \cite{Schappals2017}.
Therefore, a small amount of effort spent planning a study can pay off many times over.  In the worst case, a poorly planned study can lead to weeks or months of simulations and analyses that yield questionable results.

With this in mind, one of the objectives of this document is to provide a set of benchmark practices against which reviewers and other scientists can judge the quality of a given work.  If you read this guide in its entirety \emph{before} performing a simulation, you will have a much better sense of what constitutes (in our minds) a thoughtful simulation study.  Thus, we strongly advise that readers review and understand the concepts presented here, as well as in related reviews \cite{Grossfield2009,JCGM:GUM2008,PatroneUQreview}

%criteria applicable to your data -- and which indeed \emph{should} be applied by knowledgeable reviewers of your work.
%Thus we strongly advise understanding the concepts presented here as well as in related reviews \cite{Grossfield2009,JCGM:GUM2008}.

In a generic sense, the overall goal of a computational study is to be able to draw statistically significant conclusions regarding a particular phenomenon.  To this end, ``good statistics'' usually follow from repeated observations of a quantity-of-interest.  While such information can be obtained in a number of ways, time-series data is a natural output of many simulations and is therefore a commonly used to achieve the desired sampling.  Several observations follow.
{\color{blue} DWS Question 1: What are the ``several observations'' that follow? As far as I can tell, the only observation is that time-series data is correlated and we have to get a handle on the timescale(s) to produce uncertainty estimates.}

For one, time-series data generally displays a certain amount of autocorrelation in the sense that the numerical values of nearby points in the series tend to cluster close to one another.  Intuition dictates that correlated data does not reveal fully ``new'' information about the quantity-of-interest, and so we require uncorrelated samples to achieve meaningful sampling \cite{PatroneAIAA}.\footnote{This intuition is, strictly speaking, faulty in that anti-correlated samples actually {\it increases} our knowledge of a given random quantity relative to decorrelated samples.  See, for example, the discussion in Ref.~\cite{PatroneAIAA}.}
{\color{blue} DWS: I rewrote that footnote a bit so that it did not seem like an {\it in situ} contradiction of the main text. Please check it for technical accuracy, though.}

Thus, it is critical to ask: \emph{what are the pertinent timescales of the system?} 
Unfortunately, this question must be answered individually for each system.  You will want to study the experimental and computational literature for your particular system, although we warn that a published prior simulation of a given length does not in itself validate a new simulation of a similar or slightly increased length.  In the end, your data must be validated as well as possible by statistical analyses, such as the autocorrelation analysis described in Secs.~\ref{sec:zeroth} and \ref{sec:autocorrelation}.  Be warned that a system may possess states (regions of configuration space) that, although important, are \emph{never} visited in a given simulation set because of insufficient computational time \cite{Grossfield2009} and, furthermore, this type of error will not be discovered through the analyses presented below.
Finally, note that "system" here does not necessarily refer to a complete simulation (e.g.~a biological system with protein, solvent, ions, etc); it can also refer to some subset of the simulation for which data is desired.  For example, if one is only interested in the dynamics of a binding site in a protein, it probably is not necessary to observe the unfolding and refolding of that protein as well.

% In addition to autocorrelation analyses (Secs.\ \ref{sec:global} and \ref{sec:specific}), DMZ: seems out of place
One general strategy that will allow you to understand the relevant timescales in a system is to perform several repeats of the same simulation protocol.  As described below, repeats can be used to assess variance in \emph{any} observable, within the time you have run your simulation.
When performing simulation repeats, it is generally advised to use different starting states which are as diverse as possible; then, differences among the runs can be an indicator of inadequate sampling of the equilibrium distribution.
Alternatively, performing multiple runs from the same starting state will yield behavior particular to that starting state and potentially equilibrium information if the runs are long enough.


%And ``good statistics'' follow from repeated observations of a phenomenon, which can only happen if the simulation length \emph{exceeds} the pertinent timescales.
%PNP question: First sentence feels like it's missing something.  Specifically, we state, "the overall goal is to ...."  The overall goal of what though?  A given simulation, a given simulation study (which may involve multiple simulations), something else?  I'm confused here because I think we're mixing more generic statistics language and ideas relevant to time-series data.  The second sentence makes it sound like we're interested in getting good statistics of a time-series.  Statistically significant conclusions could apply to an estimate extracted from multiple simulations because even there, you might want "repeated observations," i.e. multiple simulations.  More generally, is this section about making sure that a single simulation is long enough or making sure that an entire simulation study has a good chance of returning meaningful predictions?

%What are the pertinent timescales?



\begin{figure}
  \centering
  % \includegraphics[width=5.8cm]{figures/1d-landscape-tslow}
  \includegraphics[width=0.9\linewidth]{figures/1d-landscape-tslow}
  \caption{
  \label{fig:landscape} 
  Schematic illustration of a free energy landscape dominated by a slow process.
  The timescales associated with a system will often reflect ``activated'' (energy-climbing) processes, although they could also indicate diffusion times for traversing a rough landscape with many small barriers.
  In the figure, the largest barrier is associated with the slowest timescale $t_{\mathrm{slow}}$, and the danger for conventional MD simulations is that the total length of the simulation may be inadequate to generate the barrier crossing.
  The presence of stochasticity implies that even a simulation as long as $t_{\mathrm{slow}}$ may not yield the key event.
  }
\end{figure}

A toy model illustrates some of these timescale-issues and their effects on sampling.
Consider the ``double-well'' free energy landscape shown in Fig.\ \ref{fig:landscape}, and note that the slowest timescale is associated with crossing the largest barrier.  Generally, you should expect that the value of \emph{any} observable (e.g., $x$ itself or another coordinate not shown or a function of those coordinates) will depend on which of the two dominant basins the system occupies.  In turn, the equilibrium average of an observable will require sampling the two basins according to their equilibrium populations.  In order to directly sample these basins, however, the length of a trajectory will have to be many times the slowest timescale, i.e. the largest barrier should be crossed multiple times.  Only in this way can the relative populations of states be inferred from time spent in each state.  Stated differently, the equilibrium populations follow from the transition rates \cite{Zuckerman2011,Chou11,Kolmogoroff1936} which can be estimated from multiple events.  For completeness, we note that there is no guarantee that sampling of a given system will be limited by a dominant barrier.  Instead, a system could exhibit a generally rough landscape with many pathways between states of interest.
Nevertheless, the same cautions apply.

What should be done if a determination is made that a system's timescales are too long for direct simulation?
The two main options would be to consider a more simplified (``coarse-grained'') model \cite{Merchant2011,Kmiecik2016} or an enhanced sampling technique (see Sec.~\ref{sec:enhanced}), bearing in mind that enhanced sampling methods are not foolproof but have their own limitations which should be considered carefully.

Lastly, whatever simulation protocol you pursue, be sure to use a well-validated piece of software [\url{https://github.com/shirtsgroup/software-physical-validation}].
If you are using your own code, check it against independent simulations on other software for a system that can be readily sampled.
%[benchmark systems???]


% !TEX root = ./main.tex

\section{Quick-and-Dirty checks that can rule out good sampling}

It is difficult to establish with certainty that good sampling has been achieved, but it is not difficult to \emph{rule out} high-quality sampling.
Here we elaborate on some quick-and-dirty tests that quickly show inadequacies in sampling.

\subsection{Zeroth-order system-wide tests}

The simplest test for poor sampling is lack of equilibration: if the system is still noticeably relaxing from its starting conformation, statistical sampling has not even begun, and thus by definition is poor.  As a result, the very first test should be to verify that the basic equilibration has occurred.  To check for this, one should inspect the time series for a number of simple scalar values, such as potential energy, system size (and area, if you are simulating a membrane or other system where one dimension is distinct from the others), temperature (if you are simulating in the NVE ensemble).  Often, simple visual inspection is sufficient to determine that the simulation is systematically changing, although more sophisticated methods have been proposed by Chodera \cite{Chodera-2016}.  If \emph{any} value appears to be systematically changing, then the system is not equilibrated.

\subsection{Tests based on configurational distance measures - e.g., RMSD for biomolecules}

\begin{wrapfigure}{r}{6cm}
  \includegraphics[width=5.8cm]{figures/rmsd/rmsd}
  \includegraphics[width=5.8cm]{figures/rmsd/rmsds}
  \caption{
  \label{f:rmsd} RMSD as a measure of convergence.  The upper panel shows the
  $\alpha$-carbon RMSD of the protein rhodopsin from its starting structure as a
  function of time.  The lower panel shows the all-to-all RMSD map computed from the same
  trajectory.  Data from Leioatts, et al \cite{Grossfield-2015}.
  }
\end{wrapfigure}

We will use the standard biomolecular RMSD (root mean-squared difference) as a generic distance measure for illustrative purposes.
Alternatives to RMSD could be a dihedral-angle distance or another measure specific to your system of interest.
Note that RMSD, like any distance in a high-dimensional space, becomes ``degenerate'' for larger values: given a reference configuration, there are a large number of configurations which differ from the reference by a given large RMSD; this is analogous to the increasing number of points in three-dimensional space with increasing radial distance from a reference point, except much worse because of the dimensionality.

Some qualitative tools for assessing global sampling based on RMSD were reviewed
in prior work \cite{Grossfield2009}.   The classic time series plot of RMSD with
respect to a crystal or other single reference structure can immediately
indicate whether the structure is still systematically changing.  Although this
kind of plot was historically used as a sampling test, it should really be
considered as another equilibration test like those discussed above.  Moreover,
it's not even a particularly good test of equilibration, because the degeneracy
of RMSD means you can't tell if the simulation is exploring new states that are
equidistant from the chosen reference.  The upper panel of Figure \ref{f:rmsd}
shows a typical curve of this sort, taken from a simulation of the G
protein-coupled receptor rhodopsin \cite{Grossfield-2015}; the curve increases
rapidly over the few nanoseconds and then roughly plateaus.  It is difficult to
assign meaning to the other features on the curve.

A better RMSD-based convergence measure is the all-to-all RMSD plot; taking the
RMSD of each snapshot in the trajectory with respect to all others allows you to
use RMSD for what it does best, identifying very similar structures.  The lower
panel of Figure \ref{f:rmsd} shows an example of this kind of plot, applied to
the same trajectory before.  By definition, all such plots have values of zero
along the diagonal, and occupation of a given state shows up as a block of
similar RMSD along the diagonal; in this case, there are 2 main states, with one
transition occuring roughly 800 ns into the trajectory.  Off diagonal ``peaks''
(regions of low RMSD between structures sampled far apart in time) indicate that
the system is revisiting previously sampled states, a necessary condition for
good statistics.  In this case, the initial state is never sampled after the
first transition, but there are a number of small transitions within the second
state.

\section{Determining and removing an equilibration or `burn-in' portion of a trajectory}
\section{Quantification of Global Sampling}

With ideal trajectory data, one would hope to be able compute arbitrary observables with reasonably small error bars.
During a simulation, it is not uncommon to monitor specific observables of interest, but after the data is obtained, it may prove necessary to compute observables not previously considered.
These points motivate the goal of estimating global sampling quality, which can be framed most simply in the context of single-trajectory data:
``Among the very large number of simulation frames (snapshots), how many are statistically independent?''
From a dynamical perspective, which also applies to Monte Carlo data, how long must one wait before the system completely loses memory of its prior configuration?
The methods noted in this section build on ideas already presented in the section on ``quick-and-dirty'' qualitative sampling analysis, but attempt to go a step further to quantify sampling quality.

\subsection{Scope and a key caveat}
The discussion here will focus largely on biomolecular systems, or more precisely, on systems for which it is straightforward to define a meaningful scalar distance between configurations.

A key caveat is needed before proceeding.
Analysis of trajectory data generally cannot make inferences about parts of configuration space not visited \cite{Grossfield2009}.
It is generally impossible to know whether configurational states absent from a trajectory are appropriately absent because they are highly improbable (extremely high energy) or because the simulation simply failed to visit them because of a high barrier or bad luck.

\subsection{Global sampling assessment for a single trajectory}
Two methods applicable for a single trajectory have been previously introduced by some of the authors, exploiting the fact that trajectory correlations are sequential.
That is, each configuration evolves from and is most similar to the immediately preceding configuration.
This picture holds for standard MD and Markov-chain MC.

Lyman and Zuckerman proposed a global ``decorrelation'' analysis by mapping a trajectory to a discretization of configuration space and analyzing the resulting statistics \cite{Lyman2007a}.
Configuration space is discretized into bins based on Voronoi cells of structurally similar configurations - e.g., based on an RMSD criterion.
The analysis method is based on the observation that the variance for any bin of a multinomial distribution is known, given a specified number of independent samples drawn from the discretized distribution.
The knowledge of the expected variance allows testing of increasing waiting times between configurations drawn from the trajectory to determine when and if the variance approaches that expected for independent samples.
The minimum waiting time yielding agreement with ideal statistics yields an estimate for the decorrelation/memory time, which implies an overall effective sample size.

A second method, employing block covariance analysis (BCOM), was presented by Romo and Grossfield \cite{Romo2011} building on ideas by Hess \cite{Hess2002}.  In essence, the method combines two standard error analysis techniques, block averaging \cite{Flyvbjerg-1989} and bootstrapping \cite{Tibshirani1998}, with a quantitative assessment of the similarity of modes determined from principal component analysis, covariance overlap \cite{Hess2002}.  The principal components are computed from subsets of the trajectory, and the similarity of the modes evaluated as a function of subset size; as the subsets get larger, the resulting modes get more similar.  This is done both for contiguous blocks of trajectory data (block averaging), and again for randomly chosen subsets of trajectory frames (bootstrapping); taking the ratio of the two values as a function of block size yields the degree of correlation in the data.  Fitting that ratio to a sum of exponentials allows one to extract the relaxation times in the sampling.  The key value of this method over others is that it implicitly takes into account the number of substates; the longest correlation time is the time required not to make a transition, but to sample a scattering of the relevant states.  This method is implemented as part of LOOS \cite{LOOS,LOOS-JCC}.

\subsection{Global sampling assessment for multiple independent trajectories}
When sampling is performed using multiple independent trajectories (whether MD or MC), additional care is required.
Analyses based solely on the assumption of sequential correlations may break down because of the unknown relationship between separate trajectories.

Zhang et al.\ extended the decorrelation/variance analysis noted above, while still retaining the basic strategy of inferring sample size based on variance \cite{Zhang2010}.
To enable assessment of multiple trajectories, the new approach focused on conformational state populations, arguing that the states fundamentally underlie equilibrium observables.
Employing a fairly simple kinetic-clustering technique to automatically define states, the approach then uses the variances in state populations among trajectories to estimate the effective sample size.

% TODO
%[ALAN, CAN BLOCK COVARIANCE BE USED EQUALLY ON MULTIPLE TRAJECTORIES?]
% Dan: probably yes, but we haven't really worked on it yet.  Instead, what I'd
% probably do is just use the coverlap between different trajectories (perhaps as
% function of sim length) to estimate similarity of sampling.  However, this is
% still a research problem, so it probably doesn't belong here.

\subsection{Global sampling assessment for enhanced sampling methods}
The family of enhanced equilibrium sampling methods, including replica exchange and variants \cite{Swendsen-1986,Sugita1999,Okamoto-2000}, metadynamics \cite{Bussi2006a,Laio2008}, adaptive biasing force \cite{Darve2001,Darve2008,Comer2015} among other methods, are complex and the resulting data may have a highly non-trivial correlation structure.
In replica exchange, for example, the ensemble at a temperature of interest will be based on multiple return visits of different sequentially correlated trajectories.

Given the subtleties of these sampling approaches, we suggest taking a 'bottom line' approach, and assessing sampling based on multiple independent runs.
The variance among these runs, if the approach is not biased, should be a measure of the overall sampling.
Hence any method applicable to multiple trajectories should be valid for analyzing multiple runs of an arbitrary method.
A caveat for the approach of Zhang et al.\ \cite{Zhang2010} is that some dynamics trajectory segments would be required to perform state construction by kinetic clustering.

% TODO
%[ALAN, CAN BLOCK COVARIANCE BE USED EQUALLY ON MULTIPLE INDEPENDENT ENHANCED SAMPLING RUNS?]
% Probably, but we haven't implemented the reweighting needed, so it doesn't belong here.

\section{Computing error in specific observables}
\label{sec:specific}

\subsection{Basics}
``What error bar should I report?''
Here we address this simple but critical question.

In general, there is no one-best practice for choosing error bars. However, in the context of simulations, we can nonetheless identify common goals when reporting such estimates: (i) to help authors and readers better understand uncertainty in data; and (ii) to provide readers with realistic information about the reproducibility of a given result.

With this in mind, we recommend the following: (a) in fields where there is a definitive standard for reporting uncertainty, the authors should follow existing conventions; (b) otherwise, such as for biomolecular simulations, \emph{authors should report (and graph) their best estimates of 90\% confidence intervals.} As explained in the glossary above, a 90\% confidence interval is a range of values that is expected to bracket 90\% of the computed predictions \emph{if statistically equivalent simulations are repeated a large number of times;} (c) when feasible, consider plotting all of the points or a histogram instead of an average with error bars.


We emphasize that as opposed to standard uncertainties (reported as a standard deviation $\sigma$), confidence intervals have several practical benefits that justify their usage. In particular, they directly quantify the statistical frequency with which we expect a given outcome, which is more relatable to everyday experience than moments of a probability distribution. As such, confidence intervals can help authors and readers better understand the implications of an uncertainty analysis. Moreover, downstream users/readers of a given paper may include less statistically-oriented readers for whom confidence intervals are a more meaningful measure of variation.

In a related vein, error bars expressed in terms of $n$ $\sigma$ can be misinterpreted as unrealistically under or overestimating uncertainty if taken at face value. For example, reporting $3$ $\sigma$ uncertainties for a normal random variable amounts to a $99.7$ \% confidence interval, which is likely to be a significant overestimate for many applications. On the other hand, $1$ $\sigma$ uncertainties only correspond to a $68$ \% confidence interval, which may be too low. Given that many readers may not take the time to make such conversions in their heads, we feel that it is safest for modelers to do this work up front.

In recommending 90 \% confidence intervals, we are admittedly attempting to address a social issue that nevertheless has important implications for science as a whole. In particular, the authors of a study and the reputation of their field do not benefit in the long run by under-representing uncertainty, since this may lead to incorrect conclusions. But perhaps just as importantly, many of the same problems can arise if uncertainties are reported in a technically correct but obscure and difficult-to-interpret manner. For example, 1 $\sigma$ error bars may not overlap and thereby mask the inability to statistically distinguish two quantities, since the corresponding confidence intervals are only 68 \%. With this in mind, we therefore wish to emphasize that visual impressions conveyed by figures in a paper are of primary importance. Regardless of what a research paper may explain carefully in text, error bars on graphs create a lasting impression and must be as informative and accurate as possible. If 90\% confidence intervals are reported, the expert reader can easily estimate the smaller standard uncertainty (especially if it is noted in the text), but showing a graph with overly small error bars is bound to mislead most readers -- even experts who do not search out the fine print.






%We frequently perform molecular simulations to make quantitative estimates of quantities like pressure, free energy or yield strain.  When reporting these estimates, it is important to also provide an estimate of the uncertainty of the result, typically as a standard error.

%A 90\% confidence interval, as explained in the glossary above, is a range of values which is expected to bracket 90\% of the observable values \emph{if the identical simulation is repeated a large number of times.}  Of course, the interval is estimated based on a single simulation, and so it is only approximate; nevertheless, as explained below, there are statistically sound means for estimating the interval from a single simulation.  If additional simulations are performed and used to estimate the confidence interval, one is then trying to estimate the confidence interval for the set of simulations -- and the corresponding interval should be smaller, in general, than that for a single simulation.

\begin{table}
    \begin{tabular}{S S}
      \toprule
       {Pressure (MPa)} & {Density (mol/L)} \\
      0.001 & 3.007(3)e-4 \\
      0.010 & 0.003011(2) \\
      0.100 & 0.03039(16) \\
      1.000 & 52.1(5) \\
      \bottomrule
    \end{tabular}
  \caption{Density as a function of pressure for water at a temperature of 400K}
  \label{tab:uncertainties}
\end{table}

[DMZ: SHOULD WE INCLUDE THIS (AND TABLE) GIVEN THAT IT SEEMS FIELD-SPECIFIC?  IF SO, WE SHOULD BE SPECIFIC ABOUT WHICH FIELDS USE THIS CONVENTION - CERTAINLY NOT BIOMOLECULAR SIMULATION.]
When reporting quantities with uncertainties, only the digits that are significant should be included and the standard error should be placed after the value in parenthesis as the uncertainty in the last digit.  In general only one digit of uncertainty need be reported unless the digit is 1, in which case, it is helpful to report two digits (in order to avoid up to 50\% roundoff error in the uncertainty).  See the \TABLE{uncertainties} for examples from hypothetical isobaric simulations of water.


\subsection{Overview of procedures for computing a confidence interval}
We remind readers that before attempting to quantify uncertainty via an error bar, the ``quick-and-dirty'' checks on sampling should be performed.
If the observable of interest is not fluctuating about a mean value but largely increasing or decreasing during the course of a simulation, it is unlikely that a reliable quantitative estimate for the observable (or an error bar) can be obtained.

For observables passing the qualitative tests noted above in Sec.\ \ref{sec:quick}, we advocate obtaining confidence intervals in one of two ways:
\begin{itemize}
\item For observables that are approximately Gaussian-distributed, an appropriately chosen \emph{coverage factor} $k$ (typically about 2, but see below) multiplying the standard uncertainty $\stdunc$ yields an estimate for a 90\% confidence interval.
\item For non-Gaussian observables, a \emph{bootstrapping} approach (described below) should be used.  An example of a potentially non-Gaussian observable is a rate-constant, which must be positive but could exhibit significant variance, so a confidence interval estimated with a coverage factor may lead to an unphysical negative lower limit.  Multimodal distributions certainly are not Gaussian.  Bootstrapping does not assume an underlying distribution but instead constructs a confidence interval based on the recorded data values, and the limits cannot fall outside the extreme data values.  Bootstrapping (or block averaging) is also necessary for cases where the observable isn't computed from a single frame, but rather from a collection of frames; this is true for quantities such as time correlation functions, and anything derived from them.
\end{itemize}

%The standard error is the standard deviation of the distribution of the results that would be obtained by repeating the simulation.  Several techniques to estimate the uncertainty are described below with varying levels of complexity and robustness.  Ultimately, a technique is robust if it can produce an uncertainty estimate that is consistent with the standard deviation of results from actually repeating the simulation and analysis of the data.

Below we describe approaches for estimating the standard uncertainty $\stdunc$ and coverage factor $k$ from a single trajectory, as well as discussing the bootstrapping approach for direct confidence-interval estimation.

Whether using a coverage factor and standard uncertainty or bootstrapping, one requires an estimate for the independent number of observations in a given simulation.  This requires care, but may be accomplished based on the effective sample size described in Sec.\ \ref{sec:global} or via block averaging, described below.  However, both methods have their limitations, and must be used with caution.  Primarily, both require the user to look at the correct quantity in order to produce the correct answer.  Block averaging will produce different effective sample sizes for different quantities; to produce robust answers, one must identify and track the slowest relevant degree of freedom in the system, which can be a non-trivial task.  Even apparently fast-varying properties may have significant statistical error if they are coupled to slower varying ones, and this error in uncertainty estimation may not be readily identifiable by solely examining the fast-varying time series.

In the absence of a reliable estimate for the number of independent observations, one can perform $n$ independent simulations and calculate the standard deviation $\sigma$ among these, yielding a standard uncertainty of $\stdunc = \sigma / \sqrt{n}$.  When computing the uncertainty with this approach, it is important to ensure that each starting configuration is also independent - or else to recognize and indicate in publication that the uncertainty refers to simulations started from a particular configuration.  The means to obtain independent starting configurations is system-dependent, but might involve repeating the protocol used to construct a configuration (solvating a protein, inserting liquid molecules in a box, etc.), using a new random seed.  However, readers are cautioned that \emph{for complex systems, it may be effectively impossible to generate truly independent starting configurations pertinent to the ensemble of interest.}  For example, a simulation of a protein in water will nearly always start from the experimental structure, which introduces some correlation in the resulting simulations even when the remaining simulation components (water, salt, etc) are regenerated de novo.

\subsection{Assessing qualitative behavior of data}

[DMZ: SINCE THIS IS A QUALITATIVE ISSUE, SHOULDN'T IT GO IN THE QUICK-AND-DIRTY SECTION?]

Generally speaking, analysis routines that extract derived quantities from raw data are often formulated on the basis of physical intuition about how that data should behave.  Before proceeding to data analysis, it is therefore useful to assess the extent to which raw data conforms to these expectations and the requirements imposed by either the modeler or the analysis routines.  Such tasks help reduce subjectivity of predictions and offer insight into when a simulation protocol should be revisited to better understand their meaningfulness \cite{patrone1}.  Unfortunately, general recipes for assessing data quality are impossible to formulate, owing to the range of physical quantities of interest to modelers.  Nonetheless, a short example will help clarify the matter.

In the context of materials science, understanding when a structural material fails is critical for many engineering applications.  In the past decade, scientists have increasingly focused on modeling the yield-strain $\epsilon_y$, which is (loosely speaking) the amount of stretching (or strain) at which it deforms irreversibly.  Intuition and experiments tells us that upon deforming a material by a fraction $1+\epsilon$, it should recover its original dimensions if $\epsilon \le \epsilon_y$ and have a residual strain $\epsilon_r = \epsilon - \epsilon_y$ if $\epsilon \ge \epsilon_y$ \cite{patrone2}.  Owing to the time-scale limitations of MD, it is also reasonable to expect that the transition in $\epsilon_r$ around yield will be smooth and not piecewise linear.  Thus, if we fit residual strain to a hyperbola, the proximity of data to the asymptotes illustrates the extent to which simulated $\epsilon_r$ conforms to the expectation that $\epsilon_r=0$ when $\epsilon < \epsilon_y$.  See Fig.~\ref{fig:yield}.

\begin{figure}
\includegraphics[width=8cm]{hyperbola.png}\caption{Yield strain tn $\epsilon_r$ as a function of applied strain $\epsilon$.  Blue $\times$ denote simulated data, whereas the smooth curve is a hyperbola fit to the data.  The green lines are asymptotes; their intersection can be taken as an estimate of $\epsilon_y$.    Bounds on yield are computed by the synthetic data method discussed in the next section.  {\it From, ``Estimation and uncertainty quantification of yield via strain recovery simulations,'' P.\ Patrone, CAMX 2016 Conference Proceedings.  Reprinted courtesy of the National Institute of Standards and Technology, U.S. Department of Commerce. Not copyrightable in the United States.}}\label{fig:yield}
\end{figure}

\subsection{Block averaging for estimating the standard uncertainty}

The standard uncertainty of a simulation observable can be estimated from the fluctuations in the quantity along with the fact that (in the absence of correlation) the squared uncertainty will be inversely proportional to the number of samples.  Block averages can be used instead of individual samples to avoid correlation in the samples while using all the data so long as the trajectory ``blocks'' (i.e., segments) are long enough to be essentially uncorrelated \cite{Friedberg1970,Flyvbjerg-1989,FrenkelSmit2002,Grossfield2009}.  The uncertainty is expressed as
\begin{equation}
  \stdunc = \sigma_{\rm blocks}/M^{1/2}
\end{equation}
where $\sigma_{\rm blocks}$ is the standard deviation of the block averages and $M$ is the number of \emph{independent} blocks.

It is crucial when performing this analysis to do so systematically as a
function of block size; as the blocks get longer, they should become independent
and the $\stdunc$ should plateau \cite{Flyvbjerg-1989,Grossfield2009}.   Another
approach is to measure the block correlation and to use it to improve the
uncertainty estimate \cite{Kolafa1986}.

\subsection{Propagation of uncertainty}

The quantities we are most interested in may not be simulation observables.  For instance, the free energy difference between two states might be measured by free energy perturbation, expressed as a function of the average of another quantity \cite{Taylor1997}.
\begin{equation}
\beta \Delta A = -\ln \left< \exp \left(-\beta \Delta U\right) \right>
\end{equation}
Although $\exp(-\beta \Delta U)$ can be measured during the simulation and its uncertainty can be estimated directly using block averages as described above, $\beta \Delta A$ cannot be handled the same way.  If we compute $\beta \Delta A$ for each block, the values will tend to take extremely positive whenever the perturbation does poorly (where $\Delta U$ is consistently large).  In the pathological case, $\Delta U$ might be $\infty$ for every sample in a block and the $\beta \Delta A=-\ln 0$ cannot be computed.


Instead of using block averages for $\beta \Delta A$, its uncertainty can be expressed as a first-order Taylor series expansion
\begin{equation}
  \sigma_{\beta \Delta A} = \sigma_{\exp(\beta \Delta U)} / \left< \exp \left(-\beta \Delta U\right) \right>
  \label{eq:propagation_bDA}
\end{equation}

Propagation of uncertainty is needed whenever the derived quantity can be expressed as a function of other random observables.  It might also be needed when the derived quantity is of a function of quantities measured in separate simulations, such as $<U(T_2)>-<U(T_1)>$.  If a derived quantity is a function of multiple observables measured within a single simulation, then terms must be included to account for the correlation between those observables.

The Taylor series approach works well in most cases and easy to use, but does have limitations.  Because this approach is based on a first-order Taylor series, propagation of uncertainty can fail in cases where a non-linear formula is used and the uncertainty is very large or the distribution of input averages is not Gaussian.  For instance, the uncertainty in $\beta \Delta A$ as prescribed by Eq.~\ref{eq:propagation_bDA} cannot exceed unity no matter how short the simulation is or how bad the sampling is.  If there is doubt as to the quality of the computed uncertainty, the uncertainty can be computed with alternative approaches such as bootstrapping to verify the Taylor series results or to identify an alternative approach that works better.

\subsection{From standard uncertainty to confidence interval for Gaussian variables}
Once a standard uncertainty value is obtained for a Gaussian-distributed random variable with mean $\langle x \rangle$, and the number of independent samples $n$ has been estimated, the 90\%-confidence interval ($\langle x \rangle -k \, \stdunc, \langle x \rangle + k \, \stdunc$) can be constructed on the basis of an established look-up table for the coverage factor $k$ based on $n$.  The theoretical basis for the table is the ``Student'' or ``$t$'' distribution, which is \emph{not} a Gaussian, but governs the behavior of an \emph{average} derived from $n$ independent Gaussian variables  \cite{JCGM:GUM2008}.  See Table \ref{tab:coveragefactors}.

When $n \leq 5$ [SHOULD THIS BE 10?], we recommend showing all data points, as a confidence interval is likely not statistically meaningful.

\begin{table}
    \begin{tabular}{S S}
      \toprule
       {$n$ (independent samples)} & {$k$ (coverage factor)} \\
       \hline
      6 & 2.02 \\
      11 & 1.81 \\
      16 & 1.75 \\
      21 &  1.72\\
      26 & 1.71 \\
      51 & 1.68 \\
      101 & 1.66 \\
      \bottomrule
    \end{tabular}
  \caption{Coverage factors $k$ required for a 90\% confidence interval for a Gaussian variable \cite{JCGM:GUM2008}.}
  \label{tab:coveragefactors}
\end{table}

As a reminder, multi-modally distributed variables with multiple peaks in their distributions cannot be considered Gaussian random variables.  Variables with a strict upper or lower limit (such as a positive-definite quantity) and long-tailed distributions likely are not Gaussian.  These cases should be treated with bootstrapping.

\subsection{Bootstrapping}

%The Taylor series approach may fail in some situations, either because a derived quantity cannot be expressed as a function of the measured simulation data or because the first-order truncation of the series is too severe.
Bootstrapping is an approach to uncertainty estimation that does not assume a particular distribution for the observable of interest or a particular kind of relationship between the observable and variables directly obtained from simulation \cite{Tibshirani1998}.  In nonparametric bootsrapping, new, ``synthetic'' data sets (corresponding to hypothetical simulation runs) are created by drawing $n$ samples (configurations) from the original collection that was generated during the actual run.  The same sample may be selected twice, while others may not be selected at all in a process called ``sampling with replacement.''  In doing so, these synthetic sets will be different even though they all have the same number of samples and draw from the same pool of data.  Having created a new set, the data is analyzed to determine the derived quantity of interest, and this process is repeated to produce multiple estimates of the quantity.  The distribution of `synthetic' observables can be directly used to construct a 90\% confidence interval from the 5\%ile to the 95\%ile value.

As with bootstrapping, the key issue is how many samples to draw from the original collection - what value of $n$ should be used?  It should be clear that a larger $n$ will yield less variation and hence a tighter confidence interval: implicitly all $n$ samples will be regarded as independent.  Hence, $n$ should be chosen to represent the number of \emph{independent} samples present in the original simulation, which can be gauged using a correlation-time method \cite{Chodera-2016,Lyman2007a} or implicitly using a block-averaging procedure -- see above.
%the uncertainty is the standard deviation of the computed quantities from the generated sets. Because the process uses original simulation samples, it does not need to make assumptions about the distribution of samples and works well even when the Taylor series approach fails.

The process described above assumes that the original simulation data is uncorrelated.  If this is not the case, then the resampling method can be reformulated in one of two ways.  The first option is to estimate the number of independent samples in the original set and to pull only that many samples to create the new data sets.  The second option is to group the samples into blocks that are uncorrelated and to then use the block averages as the samples for bootstrapping.

Alternatively, one could use the difference between errors estimated via block averaging and bootstrapping as a measure of the correlation; if one tracks the bootstrapped and block averaged estimates of a quantity's uncertainty as a function of block size, the only difference between the two modes of calculation is whether the data is correlated.  The decay in the ratio of the two quantities as a function of time is a measure of the correlation time in the sample \cite{Romo2011}.

An alternate approach that can directly account for correlations is called parametric bootstrapping.  The main idea behind this method is to model the original data as a deterministic function (which can be zero, constant, or have free parameters) plus additive noise.  The parameters of this model, including the structure of the noise (i.e.\ its covariance), can be determined through a statistical inference procedure.  Having calibrated the model, random number generators can be used to sample the noise, which is then added back to the trial function to generate a synthetic data set.   As with the nonparametric bootstrap, the generated data can be used to compute the derived quantity of interest, and the uncertainty can be obtained from the statistics of the values compute with different generated sets.

To further clarify the procedure of parametric boostrapping, consider the simplest case in which the data is a collection of uncorrelated random variables fluctuating about a constant mean.  In this situation, one could estimate (I) the deterministic part of a parametric model using the sample mean $\mu$ of the data, and (II) the stochastic part as a Gaussian random variable whose variance equals the sample variance.  If instead the data are correlated (e.g.\ as in a time-series of simulated observables),  one can postulate a covariance function to describe the structure of this randomness.  Often these covariance functions are formulated with free parameters (often called ``hyperparameters'') that characterize properties such as the noise-scale and characteristic length of correlations \cite{Rasmussen}.  In such cases, determining the hyperparameters may require more sophisticated techinques such as maximum likelihood analyses or Bayesian approaches; see, for example, Ref.~\cite{Rasmussen}.  See also Refs.~\cite{patrone1,patrone2,patrone3,Boettinger2017} for examples and practical implementations applied to cases in which the determinsitic component of the data is not constant.

It is important to note that various bootstrapping approaches can and often are used as uncertainty propagation tools.  Nonetheless, care should be exercised when using such methods with nonlinear functions.  In the free energy example, setting $\langle \exp(-\beta \delta U)\rangle = 1 \pm 0.5$ and generating new estimates from a Gaussian centered at 1 with a width of 0.5 will eventually output negative numbers, which is mathematically nonsensical and problematic for any function that takes strictly non-negative inputs.  Thus, one should be aware of any distributional assumptions imposed either by the physics of the problem or the analyses of synthetic data.


\subsection{Dark uncertainty analyses}
%- ‘Dark uncertainty’ analysis [Paul]

In some cases, multiple simulations of the same physical observable $\tau$ may yield predictions whose error bars do not overlap.  This situation can arise, for example, in simulations of the glass transition temperature when undersampling the crosslinked network structure of certain polymers.  In such cases, it is reasonable to postulate an unaccounted for source of uncertainty, which we colorfully refer to as ``dark uncertainty.''  In the context of a statistical model, we postulate that the probability of a simulation output depends on the unobserved or ``true'' mean value $\bar \tau$, an uncertainty $\sigma_i^2$ whose value is specific to the simulation (estimated, e.g.\ according to uncertainty propagation), and the unaccounted-for dark uncertinty $y^2$.  (For simplicity, the $\sigma_i^2$ and $y^2$ should be treated as variances.)

While details are beyond this scope of this document, such a model motivates an estimate of $\bar \tau$ of the form
\begin{align}
\bar \tau \approx \mathcal T \propto \sum_i \frac{T_i}{\sigma_i^2 + y^2}, \label{eq:darkmean}
\end{align}
where $T_i$ is the prediction from the $i$th simulation, $\sigma_i^2$ is its associated ``within-simulation'' uncertainty, and $y^2$ is the dark or between-simulation uncertainty; note that the latter does not depend on $i$.  The variable $y^2$ can be estimated from a maximum-likelihood analysis of the data and amounts to numerically solving a relatively simple nonlinear equation (see Ref.~\cite{patrone1}).  Equation~\eqref{eq:darkmean} is useful insofar as it weights simulated results according to their certainty while reducing the impact of overconfident predictions (e.g. having small $\sigma_i^2$).  Additional details on this method are provided in Ref.~\cite{patrone1} and the references contained therein.



- Basics: how to report, what goal to shoot for, significant figures
- When should you not trust uncertainties
    - Unknown unknowns
- If calculating a derived quantity, consider error in conversion from raw data
    - Propagation of error (this doesn’t mean publishing your results!)
    - Taylor series expansion can handle cases where derived quantity is a direct function of measured data
    - Wikipedia: Propagation of uncertainty
    - Bootstrapping [Andrew/Dan S]
used for cases where the derived quantity is not simple function of the measured data.
An Introduction to the Bootstrap
    - Need to know correlation to correctly estimate sample size
    - Otherwise just gives relative uncertainty
- Correlation time analysis [Dan Z]
- Block averaging [Dan Z/Alan?/Dan S] Flyvjberg and Petersen
- MOST OF THESE ALGORITHMS FAIL IF THE TRAJECTORY IS WAY TOO SHORT
    - If you miss the timescale by enough, you can’t tell
    - YOU HAVE TO THINK ABOUT THIS IN ADVANCE
- Link out to transport doc

\section{Assessing Uncertainty in Enhanced Sampling Simulations}
\label{sec:enhanced}

While recent advances in computational hardware have allowed MD simulations of systems with biological relevance to routinely reach timescales ranging from hundreds of \si{\nano\second} to \si{\micro\second}, in many cases this is still not long enough to obtain equilibrated (i.e. Boltzmann-weighted) structural populations.  Intrinsic timescales of the systems may be much longer.
Enhanced sampling methods can be used to obtain well-converged ensembles faster than conventional MD. In general, enhanced sampling methods work through a combination of modifying the underlying energy landscape and/or thermodynamics parameters to increase the rate at which energy barriers are crossed along with some form of reweighting to recover the unbiased ensemble \cite{Zuckerman2011}. However, such methods do not guarantee a converged ensemble, and care must be taken when using and evaluating enhanced sampling methods.

Generally speaking, uncertainty analysis is more challenging for data generated by an enhanced sampling method.
The family of enhanced equilibrium sampling methods, including replica exchange and variants \cite{Swendsen-1986,Sugita1999,Okamoto-2000}, metadynamics \cite{Bussi2006a,Laio2008}, adaptive biasing force \cite{Darve2001,Darve2008,Comer2015} among other methods, are complex and the resulting data may have a highly non-trivial correlation structure.
In replica exchange, for example, the ensemble at a temperature of interest will be based on multiple return visits of different sequentially correlated trajectories.

Before performing an enhanceds-sampling simulation, consider carefully whether the technique is needed, and consult the literature for best practices in setting up a simulation.
Even a straightforward MD simulation requires considerable planning, and the complexity is much greater for enhanced techniques.

Given the subtleties of these sampling approaches, when possible, consider taking a ``bottom line'' approach, and assessing sampling based on multiple independent runs.
The variance among these runs, if the approach is not biased, will help to quantify the overall sampling.
Note that methods applicable to global assessment of multiple trajectories (Sec.\ \ref{sec:globalMultiTraj}) should be valid for analyzing multiple runs of an arbitrary method.
However, a caveat for the approach of Zhang et al.\ \cite{Zhang2010} is that some dynamics trajectory segments would be required to perform state construction by kinetic clustering.

\subsection{Replica Exchange Molecular Dynamics}
One of the most popular enhanced sampling methods is replica exchange MD (REMD) \citep{Sugita1999}; see also \cite{Swendsen-1986}. Broadly speaking, REMD consists of running parallel MD simulations on a number of non-interacting replicas of a system, each with a different Hamiltonian and/or thermodynamics parameters (e.g., temperature), and periodically exchanging system coordinates between replicas according to a Metropolis criterion which maintains Boltzmann-factor sampling for all replicas.  
%\textcolor{red}{PNP comment: I think we need a sentence or two explaining this metropolis criterion.}

%PNP question: I suspect that REMD doesn't actually change the Hamiltonian, but I could be wrong.  Are energy barriers actually modified, and if so, is it through a force-field modification?  I always thought replica exchange only changes temperatures.  Also, the notion of different replicates having different Hamiltonians means they can't be replicas representative of the same system, correct?


In order to assess the results of a REMD simulation, it is important to consider not just the overall convergence of the simulation to the correct Boltzmann-weighted ensemble of structures (via combined clustering, combined PC projection overlap analysis, etc.), but how efficiently the REMD simulation is doing so. These concepts are termed "thermodynamic efficiency" and "mixing efficiency" by Abraham and Gready,\citep{Abraham2008} and it is quite possible to achieve one without the other; both must be assessed. In order for sampling to be efficient, coordinates must be able to move freely in replica space.

In practical settings, several metrics are often used to assess these two efficiencies, a few of which we list below.  In these definitions, note that we refer to both "coordinate trajectories" and "replica trajectories". A "coordinate trajectory" follows an individual system's continuous trajectory as it traverses replica space (e.g., a system experiencing multiple temperatures as it is exchanged during a temperature REMD simulation). A "replica trajectory" is the sequence of configurations corresponding to a single replica under fixed Hamiltonian and thermodynamic conditions, (e.g., all structures at a temperature of 300 K in a temperature REMD simulation).  Thus, a replica trajectory consists of concatenated coordinate-trajectory segments and \textit{vice versa}.

Below are several checks that should be applied to REMD simulation data.

\begin{itemize}
  \item Exchange acceptance. The exchange acceptance rate (i.e. the number of exchanges divided by the number of exchange attempts) between neighboring replicas should be roughly equivalent to each other and to the target acceptance rate. A low exchange acceptance between neighboring replicas relative to the average exchange acceptance rate creates a bottleneck in replica space which in turn can lead to poor sampling of the overall configuration space.  In such cases, the replica spacing may need to be decreased or additional replicas used.  Conversely, a high exchange acceptance rate between neighboring replicas relative to the average exchange acceptance rate may indicate that more resources than necessary are being used to simulate replicates, and that good sampling can be achieved with fewer replicas or larger replica spacing.  
%\textcolor{red}{PNP comment: One problem with this definition is that ``high'' and ``low'' are not really defined.  I suspect these are user-defined ideas, but do they bear any relation to the other UQ we have discussed?}
  \item Replica round-trips. The time taken for a coordinate trajectory to travel from the lowest replica to the highest and back is called the replica "round trip" time. Over the course of a REMD simulation, any given coordinate trajectory should make multiple round trips. The rationale behind this is that every replica should contribute to enhancing the sampling of every set of starting coordinates.  
%\textcolor{red}{PNP question: why should any trajectory make multiple round trips?} 
One can look at the average, minimum, and maximum round trip times among the coordinate trajectories: these should be similar for any given set of coordinates. See e.g. Figure 6 in \citep{Roe2014}. %\textcolor{red}{PNP comment: not clear to me why these should be equivalent.  Isn't there a distribution of times?} 
If they are not, it is likely due to one or more bottlenecks in replica space which can be identified by a relatively low exchange acceptance rate (see the previous bullet point).
  \item Replica residence time. The time a coordinate trajectory spends at a replica is called the "replica residence time". For replica sampling to be efficient, the replica residence time for each set of starting coordinates at each replica should be roughly equivalent. If it is not (i.e. if a set of starting coordinates is spending a much larger amount of time at certain replicas compared to the overall average) this can also indicate one of more bottlenecks in replica space. An example of this is shown in Figure 7 in Roe et al. \citep{Roe2014}. 
  %\textcolor{red}{Can you elaborate briefly?  What does equivalent mean in this context (same order of magnitude?) and why should they be equivalent?}
  \item Distributions of quantities calculated from coordinate trajectories. If all coordinates are moving freely in replica space, they should eventually converge to the same ensemble of structures. Comparing distributions of various quantities from coordinate trajectories can provide a measure of how converged the simulation is. For example, one can compare the distribution of RMSD values of coordinate trajectories to a common reference structure; see e.g. Figure 8 in Henriksen et al..\citep{Henriksen2013} Poor overlap can be an indication that replica efficiency is poor or the simulation is not yet converged.
\end{itemize}

All of the above quantities (replica residence time, round trip time, lifetimes etc) can be calculated with CPPTRAJ,\citep{Roe2013} which is freely available from \url{https://github.com/Amber-MD/cpptraj} or as part of AmberTools (\url{http://ambermd.org}).

It may also be useful to perform multiple REMD runs.  Using the standard uncertainty among runs can quantify uncertainty and provide the basis for a confidence interval with an appropriate coverage factor - see definitions in Sec.\ \ref{sec:scope}.  If the ensembles produced depend significantly on the set of starting configurations, that is a sign of incomplete sampling.

\subsection{Weighted Ensemble simulations}

The weighted ensemble (WE) method orchestrates an ensemble of trajectories that are intermittently pruned or replicated in order to enhance sampling of difficult-to-access regions of configuration space \cite{Huber-1996}.
The final set of trajectories can be visualized as a tree structure based on the occasional replication and pruning events.
WE is an unbiased method that can be used to sample rare transient behavior \cite{Zhang2010a} as well as steady states \cite{Bhatt2010a} including equilibrium \cite{Suarez2014}.

Like other enhanced sampling methods, WE's tree of trajectories has a complex correlation structure requiring care for uncertainty analysis.
It is important to understand the basic theory and limitations of the WE method, as is discussed in a
\href{https://westpa.github.io/westpa/overview.html}{WE overview document}.

From a practical standpoint, the safest way to assess uncertainty in WE simulations is to run multiple instances (which can be seeded from identical or different starting structures depending on the desired calculation) from which a variance and standard uncertainty in any observable can be calculated.
Note particularly that WE tracks the time evolution of observables as the system relaxes (perhaps quite slowly) to equilibrium or another steady state \cite{Zhang2010a}; hence, the variance computed in an observable from multiple runs should be based on values at the same time point.

When it is necessary to estimate uncertainty based on a single WE run, the user should treat the (ensemble-weighted) value of an observable measured over time much like an observable in a standard single MD simulation; this is because the correlations in ensemble averages are sequential in time.
First, as discussed in Sec.\ \ref{sec:quick}, the time trace of the observable should be inspected for relaxation to a nearly constant value about which fluctuations occur.
A transient/equilibration period should be removed in analogy to MD - see Sec.\ \ref{sec:equil} - and then best practices for single obervable uncertainties should be followed as described in Sec.\ \ref{sec:specific}.
Despite this rather neat analogy to conventional MD, experience has shown that run-to-run variance in WE simulations of challenging systems can be  large, so multiple runs are  advised.  In the future, variance-reduction techniques may alleviate the need for multiple runs.



\section{Acknowledgments}
%Funder and other information can be given here.
The authors appreciate helpful discussions with Pascal T. Merz.
DMZ acknowledges support from NIH Grant GM115805.


\bibliography{refs}
%\bibliographystyle{vancouver-livecoms}

\end{document}
