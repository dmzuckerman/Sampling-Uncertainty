\subsection{Key Definitions [to be refined]}
The reader should be familiar with a set of basic statistical and simulation concepts.
[and we should provide links/refs for each.]

\cite{JCGM:GUM2008,JCGM:VIM2012}

Glossary of Statistical Terms:
\begin{itemize}
%terms added by PNP and DWS
\item {\bf Standard Uncertainty}: Uncertainty in a result as expressed in terms of a standard deviation.

\item {\bf Arithmetic mean}: An estimate of the expectation value of a random quantity. The arithmetic mean is given by the formula
  %
  \begin{equation}
    \bar{q} = \dfrac{1}{n} \sum_{k=1}^{n} q_k
  \end{equation}
  %
  where $q_j$ is an experimental realization of the random variable and $n$ is the number of samples.
\item {\bf Experimental standard deviation}: An estimate of the standard deviation of a random variable, given by the formula
  % 
  \begin{equation}
    s\left(q_k\right) = \sqrt{\dfrac{\sum_{j=1}^n\left(q_j - \bar{q}\right)^2}{n-1}}
  \end{equation}
  %
  where $q_j$, $\bar{q}$, and $n$ are as defined previously.
\item {\bf Experimental standard deviation of the mean}: An estimate of the standard deviation of the distribution of arithmetic mean, given by the formula
  % 
  \begin{equation}
    s\left(\bar{q}\right) = \dfrac{s\left(q_k\right)}{\sqrt{n}},
  \end{equation}
  %
  and is used to characterize the dispersion of the arithmetic mean relative to the expectation value of the same quantity.


\item {\bf Derived observables}: Quantities derived from `non-trivial' analyses of raw data, such as free energies.

\item {\bf Accuracy}: The degree of agreement with a reference value, which may be an experimental measurement or the result of a well-sampled simulation.

\item {\bf Correlation time}: The time over which samples/configurations in a MC or MD trajectory retain some "memory" of one another, since each configuration is generated from the preceding one.  Roughly, the total simulation time divided by the (longest) correlation time gives an estimate of the number of \emph{independent} samples that will govern overall statistical quality of the data.

\item {\bf Confidence interval}: A statistically derived pair of lower and upper bounds between which the expectation value of a random quantity is likely to fall, as quantified by a percentage.

\item {\bf Raw data}: The numbers that the computer program directly generates as it runs -- typically configurations, and also velocities in molecular dynamics.

\item {\bf Precision}: The amount of variability in an estimate (based on repeating a given simulation protocol multiple times).
      Better sampling in an individual simulation leads to higher precision..
  
% Older versions
\item Precision: The amount of variability in an estimate (based on repeating a given simulation protocol multiple times).
      Better sampling in an individual simulation leads to higher precision.
      The standard error of the mean is usually the key measure of the \emph{scale} of the statistical uncertainty - i.e., precision.
    \item Confidence Interval: A statistically derived pair of minimum and maximum values within which the mean of an observable is likely to fall, as quantified by a percentage.  Note that useful confidence intervals (e.g., 90 or 95\%) tend to be roughly \emph{four times} the standard error of the mean (from minimum to maximum).
    \item Accuracy: The degree of agreement with a reference value, which may be an experimental measurement or the result of a well-sampled simulation.
    \item Raw data: The numbers that the computer program directly generates as it runs -- typically configurations, and also velocities in molecular dynamics.
    \item Derived observables:  Quantities derived from `non-trivial' analyses of raw data, such as free energies.
    \item Correlation time: The time over which samples/configurations in a MC or MD trajectory retain some "memory" of one another, since each configuration is generated from the preceding one.  Roughly, the total simulation time divided by the (longest) correlation time gives an estimate of the number of \emph{independent} samples that will govern overall statistical quality of the data.

\end{itemize}

Table of equivalencies

Arithmetic mean = ``sample mean''\\
Experimental standard deviation: ``sample standard deviation''\\
Experimental standard deviation of the mean: ``standard error''\\

