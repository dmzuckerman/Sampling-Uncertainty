\section{Quick-and-Dirty checks that can rule out good sampling}

It is difficult to establish with certainty that good sampling has been achieved, but it is not difficult to \emph{rule out} high-quality sampling.
Here we elaborate on some quick-and-dirty tests that quickly show inadequacies in sampling.

\subsection{Tests based on configurational distance measures - e.g., RMSD for biomolecules}

We will use the standard biomolecular RMSD (root mean-squared difference) as a generic distance measure for illustrative purposes.
Alternatives to RMSD could be a dihedral-angle distance or another measure specific to your system of interest.
Note that RMSD, like any distance in a high-dimensional space, becomes ``degenerate'' for larger values: given a reference configuration, there are a large number of configurations which differ from the reference by a given large RMSD; this is analogous to the increasing number of points in three-dimensional space with increasing radial distance from a reference point, except much worse because of the dimensionality. 

Some qualitative tools for assessing global sampling based on RMSD were reviewed in prior work \cite{Grossfield2009}.  
The classic plot of RMSD (with respect to a crystal or other single reference structure) vs.\ time (or MC steps) can give an immediate visual of the trajectory.  
If the the RMSD value has not leveled off and fluctuated multiple times around some mean value, it is unlikely that true global equilibrium sampling has been achieved.  
A generalization of this approach is to plot in greyscale the pairwise RMSD values for equally spaced frames of the trajectory.  
In such a plot, one would like to see that different configurational states have been visited multiple times - i.e., low RMSD values repeatedly recur for most of the configurations.  
This is a fairly strict, albeit qualitative, test.
